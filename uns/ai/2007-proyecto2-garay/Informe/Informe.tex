\documentclass[a4paper,12pt]{report}

\oddsidemargin -1.4mm
\evensidemargin -1.4mm
\textwidth 170mm
\textheight 232mm

\title{Proyecto 2 \\ Inteligencia Artificial}
\author{Garay, I\~{n}aki (L.U. 67387)}
\date{Segundo Cuatrimestre 2007}

\begin{document}

\maketitle
\tableofcontents

\chapter{Archivos}
La entrega del proyecto consta de los siguientes archivos:

\

\textbf{Prolog:}
\begin{enumerate}
\item \texttt{./prolog/comarca.pl}: Predicados de los simuladores de entorno.
\item \texttt{./prolog/agentedfs.pl}: Predicados del agente DFS.
\item \texttt{./prolog/agenteastar.pl}: Predicados del agente $A^{*}$.
\item \texttt{./prolog/search.pl}: Predicados comunes a ambas estrategias de b\'{u}squeda.
\item \texttt{./prolog/agentpred.pl}: Predicados comunes a ambos agentes.
\item \texttt{./prolog/outputpred.pl}: Predicados de utilidad poro formatear salida.
\item \texttt{./prolog/defworld.pl}: Hechos definiendo la comarca.
\end{enumerate}

\textbf{Java:}
\begin{enumerate}
\item \texttt{./java/Proyecto2/}: NetBeans Project Folder de la interfaz gr\'{a}fica.
\item \texttt{./java/Proyecto2/src/Proyecto2/Proyecto2.java}:
\end{enumerate}

\textbf{Documentaci\'{o}n:}
\begin{enumerate}
\item \texttt{./Informe/Informe.pdf}: Este informe.
\item \texttt{./Informe/Informe.tex}: C\'{o}digo \LaTeX\ de este informe.
\item \texttt{./Informe/predicate\_invocation\_hierarchy.bmp}: Gr\'{a}fico ilustrando las relaciones entre predicados
implementados.
\end{enumerate}

\chapter{Java}

\section{Instrucciones de Uso}

Se recomienda copiar la carpeta ``\texttt{Test}'' al disco rigido, ya que la interfaz crea el archivo
\texttt{newworld.pl} en el directorio de ejecucion.

La interfaz presenta un menu principal, dos pesta\~{n}as (\textit{``World''} y \textit{``Agent Map''}), un panel con un
bot\'{o}n (\textit{``Replay''}) y un \'{a}rea de texto.
El menu principal contiene un solo menu (\textit{``File''}), el cual contiene tres items: \textit{``New Simulation''},
\textit{``Run Simulation''}, y \textit{``Exit''}.

\

El item \textit{``New Simulation''} inicia una simulaci\'{o}n nueva, mostrando un di\'{a}logo para crear la nueva
comarca.
El di\'{a}logo permite elegir el tama\~{n}o de la comarca, y contiene un \textit{CheckBox} rotulado \textit{``Default
World''}.
Si el \textit{CheckBox} esta seleccionado, los campos para el ingreso de la cantidad de filas y columnas son
deshabilitados, y se crear\'{a} la comarca presentada en el enunciado del proyecto.
Si el \textit{CheckBox} no esta seleccionado, al pulsar el bot\'{o}n \textit{``Create World''} se crear\'{a} una comarca
``en blanco'' del tama\~{n}o especificado, las celdas inicializadas como pasto sin contenido, con la excepci\'{o}n de las
celdas de los bordes, que ser\'{a} bosques.
Una vez pulsado el bot\'{o}n \textit{``Create World''}, se cerrar\'{a} el di\'{a}logo, se crear\'{a} la comarca y se
habilitar\'{a} el item \textit{``Run Simulation''}.

\

El item \textit{``Run Simulation''} corre una simulaci\'{o}n una vez creada la comarca.
Como se mencion\'{o}, permanece deshabilitado mientras no se haya creado una comarca.
Este item realiza las consultas al motor \texttt{Prolog}.
Una vez corrida la simulaci\'{o}n, se habilitar\'{a} el bot\'{o}n \textit{``Replay''} en la ventana principal.

\

\textit{``Exit''} cierra la interface.

\

Las dos pesta\~{n}as \textit{``World''} y \textit{``Agent Map''} corresponden a la comarca y al mapa de la comarca del
agente, respectivamente.

\

Una vez creado el mundo, corrida la simulaci\'{o}n y habilitado el bot\'{o}n \textit{``Replay''}, este servir\'{a} para
mostrar los resultados.
Cada vez que se pulse el bot\'{o}n \textit{``Replay''}, se actualizar\'{a} la comarca y se mostrar\'{a} una iteraci\'{o}n
de la simulaci\'{o}n.
Las celdas que pertenecen al camino actual del agente hacia el castillo apareceran remarcadas en negro.
Las celdas que pertenecen a la frontera apareceran remarcadas en rojo.

\

El \'{a}rea de texto muestra las consultas hechas al int\'{e}rprete Prolog y mensajes varios.

\section{Edici\'{o}n de la comarca}

Si se desea cambiar la comarca antes de correr la simulaci\'{o}n, se puede pulsar con el bot\'{o}n derecho del mouse
sobre la celda que se desea modificar, y aparecer\'{a} un menu popup con las opciones correspondientes.
El terreno siempre se puede modificar, pero los contenidos habilitados est\'{a}n sujetos al terreno de la celda.
Las celdas de los bordes tienen el menu deshabilitado, por lo que no se pueden modificar.

\

En el caso que se desee colocar un castillo cuando ya hay uno definido en la comarca, no habr\'{a} cambios.
Si se desea cambiar de lugar el castillo, primero habr\'{a} que eliminar el castillo existente y luego colocarlo en la
posici\'{o}n nueva.
Esta caracter\'{i}stica se implement\'{o} para evitar que se generase un mapa con dos castillos.

\

Es responsabilidad del usuario no ingresar configuraciones de mapa invalidas que incluyan puentes consecutivos, o
posiciones iniciales del agente invalidas, como un bosque o agua.

\

Cambios a la comarca despues de correr la simulaci\'{o}n no estan contemplados y pueden causar comportamiento
impredecible y/o inestable.

\chapter{Decisiones de Dise\~{n}o}

La decisi\'{o}n de dise\~{n}o m\'{a}s importante tomada fue la de minimizar la interacci\'{o}n Java-Prolog, dise\~{n}ando
en base a esto el entorno de simulaci\'{o}n no interactivo y el esquema de uso de objetos de clase \textit{Query} para
obtener un ``historial'' de la simulaci\'{o}n una vaz acabada, en lugar de actualizar la pantalla en cada iteraci\'{o}n
de la misma.

\

Otra decisi\'{o}n de dise\~{n}o importante fue no duplicar el c\'{o}digo Prolog de los dos agentes, ya que solamente
difer\'{i}an en la estrategia de b\'{u}squeda usada.
De esta manera, se parametriz\'{o} el nombre del agente en el predicado que encapsulaba la inteligencia
(\texttt{choose\_\_action/3}), de modo que \'{e}ste, en los casos necesarios, se basase en este par\'{a}metro para elegir
la estrategia de b\'{u}squeda.

\

Otra decision de dise\~{n}o importante fue tomar el hecho de buscar un camino a la meta como una acci\'{o}n expl\'{i}cita
en la l\'{o}gica del agente.
El agente inicia una b\'{u}squeda en dos ocasiones: cuando comienza la simulaci\'{o}n, y cuando el camino actual no es
viable (como por ejemplo cuando se encuentra con un puente roto donde supuso uno sano).

\

Una decisi\'{o}n de dise\~{n}o menor fue el uso de menues popup para facilitar la edici\'{o}n de la comarca.

\chapter{Ejemplos de Corridas}

Desde la interfaz:

Seleccionar ``Menu'', ``New Simulation'', elegir opciones por defecto y pulsar ``Create World''.
A continuacion ir a ``Menu'', ``Run'', elegir posicion inicial por defecto y el agente \texttt{dfs},
y pulsar ``Run Simulation''. Pulsar repetidamente el boton ``Replay''.

\

Idem anterior pero con el agente astar.

\

Idem dos anteriores pero cambiando posicion inicial (recordar ingresar una posicion inicial valida).

\

Seleccionar ``Menu'', ``New Simulation'', destildar ``Default World'', y colocar un castillo. A continuacion correr la
simulacion cont ambos agentes.

\

Se pueden probar casos patologicos como una comarca dividida en dos por un rio sin puentes. O un agente atrapado en una
isla, etc. En estos casos el agente abandona por falta de camino.

\

Se recomiendan las siguientes consultas desde el interprete Prolog:

Consultar \texttt{comarca.pl} y a continuacion \texttt{defworld.pl}.
\begin{verbatim}
:- go([10,10],dfs).
:- go([10,10],astar).
\end{verbatim}

\chapter{Prolog}

La parte en Prolog del proyecto consiste en tres subpartes fuertemente interrelacionadas, pero claramente
distinguibles: los entornos de simulaci\'{o}n, los agentes y las estrategias de b\'{u}squeda.

\

Los entornos de simulaci\'{o}n y el punto de entrada a la l\'{o}gica del programa se encuentra en el archivo
\texttt{comarca.pl}.

\

Predicados de utilidad que formatean y presentan datos de salida se encuentran en \texttt{outputpred.pl}.
Estos predicados no se documentar\'{a}n.

\

Los archivos \texttt{agentedfs.pl} y \texttt{agenteastar.pl} definen los predicados de los dos agentes.
Estos archivos son triviales debido a que los agentes comparten la misma l\'{o}gica y se parametriza el nombre del agente
para las situaciones en que difieren sus acciones, \textit{i.e.}, en el momento de elegir la estrategia de b\'{u}squeda.

\

El archivo \texttt{agentpred.pl} contiene los predicados comunes a los dos agentes, que constituye el cuerpo de su
l\'{o}gica.

\

El archivo \texttt{search.pl} contiene los predicados de ambas estrategias de b\'{u}squeda.

\

El archivo \texttt{defworld.pl} contiene los hechos que definen la comarca presentada en el enunciado (el ``default
world''). Esta representaci\'{o}n fue escrita manualmente, y si se elige la opci\'{o}n de generar el mundo por defecto en
la interfaz gr\'{a}fica, el archivo generado (\texttt{newworld.pl}) ser\'{a} id\'{e}ntico.

\

Se incluye un gr\'{a}fico que ilustra la jerarqu\'{i}a de invocaci\'{o}n de los predicados, y se los agrupo segun su
prop\'{o}sito en el programa. Se espera que sea de utilidad para la r\'{a}pida comprensi\'{o}n a nivel abstracto de la
estructura del c\'{o}digo implementado.

\newpage

\

\newpage

\section{Entornos de Simulaci\'{o}n}

Se implementaron dos entornos de simulaci\'{o}n: uno (\texttt{start\_\_simulation/4}) que al llamar su predicado de
entrada devuelve una lista con un registro de las acciones del agente en cada iteraci\'{o}n de la simulaci\'{o}n (y otros
datos relevantes), y otro simulador interactivo \newline (\texttt{start\_\_interactive\_\_simulation/3}), que en lugar de
correr toda la simulaci\'{o}n inmediatamente, en cada iteraci\'{o}n muestra el estado del entorno, la posici\'{o}n del
agente dentro del mundo, su percepci\'{o}n y la frontera actual, y espera que el usuario pulse una tecla para pasar a la
siguiente iteraci\'{o}n.
El simulador interactivo esta pensado para correrse desde el interprete Prolog, mientras que el otro simulador es el que
utiliza la interfaz Java.

\

Ambos entornos tienen la misma estructura b\'{a}sica, la diferencia principal es que uno devuelve los datos en una lista,
y el otro los presenta por pantalla.
Ambos tienen un predicado de inicializaci\'{o}n, y otro que constituye el bucle de iteraci\'{o}n.
Ambos predicados de inicializaci\'{o}n llaman a sus respectivos bucles mediante \texttt{once/1}, de modo que siempre dan
una sola simulaci\'{o}n

\

A continuaci\'{o}n se describir\'{a}n los predicados m\'{a}s importantes de cada archivo.

\subsection{\texttt{comarca.pl}}

\texttt{start\_\_simulation/4} es el punto de entrada al simulador no interactivo.
Su primer argumento es la posicion inicial del agente, el segundo el nombre del agente el tercero devuelve la raz\'{o}n
por la cual termin\'{o} la simulaci\'{o}n y el cuarto devuelve una lista de listas.
Cada elemento de esta lista corresponde a una iteraci\'{o}n del simulador, y contiene la posici\'{o}n, el camino que
faltaba recorrer, el costo del camino encontrado, y las celdas frontera en ese momento.

Su proposito principal es inicializar el entorno, limpiando la base de datos de informaci\'{o}n que pudiese acumularse en
simulaciones anteriores, retractando y asertando valores de estado iniciales.

\begin{verbatim}
% ENVIRONMENT INTERNAL STATE

:- dynamic agent_position/1.     % Maintains the agent's current position
:- dynamic agents_last_action/1. % Maintains the agent's last action
:- dynamic next_percept/1.       % Maintains the agent's next percept

%%%%%%%%%%%%%%%%%%%%%%%%%%%%%%%%%%%%%%%%%%%%%%%%%%%%%%%%%%%%%%%%%%%%%%%%%%%%%%%%
% GENERAL ENVIRONMENT SIMULATOR

% Starts the simulation, running Agent from Initial_Position
% Initial_Position is a two-element list, first element indicates row
% component of the initial position, second element indicates column component of
% the initial position.
start_simulation(Initial_Position, Agent, Termination_Reason, Simulation_Record) :-

    % Clear and initialize the environment internal state
    retractall( agent_position(_)                ),
    retractall( agents_last_action(_)            ),
    retractall( next_percept(_)                  ),
    retractall( path(_, _, _)                    ),
    retractall( camouflage(_)                    ),
    asserta(    agent_position(Initial_Position) ),
    asserta(    agents_last_action(none)         ),
    asserta(    next_percept(init)               ),
    asserta(    path([], 0, [])                  ),
    asserta(    camouflage(off)                  ),

    % Run the simulation
    once(run_environment(
            update_state,
            Agent,
            termination_function(Termination_Reason),
            Simulation_Record)).

%%%%%%%%%%%%%%%%%%%%%%%%%%%%%%%%%%%%%%%%%%%%%%%%%%%%%%%%%%%%%%%%%%%%%%%%%%%%%%%%
run_environment(_Update_Function, _Agent, Termination_Function, []) :-
    call(Termination_Function).

run_environment(
        Update_Function,
        Agent,
        Termination_Function,
        [[Position,
          Action,
          Current_Path,
          Current_Paths_Cost,
          Current_Positions_Frontier] | Simulation_Record]) :-

    get_percept(P),

    % create and call the agent predicate
    A =.. [Agent, P, Action],
    call(A),

    % create and call the update predicate
    U =.. [Update_Function, Action],
    call(U),

    % get position and path data to store in the simulation history
    agent_position(Position),
    (
     path(Current_Path, Current_Paths_Cost, [Current_Positions_Frontier|_]) ;
     path(Current_Path, Current_Paths_Cost, Current_Positions_Frontier)
    ),

    run_environment(Update_Function, Agent, Termination_Function, Simulation_Record).

\end{verbatim}

Este es el simulador de entorno interactivo.
El predicado \texttt{go/2} es el requerido por el enunciado.

\begin{verbatim}
%%%%%%%%%%%%%%%%%%%%%%%%%%%%%%%%%%%%%%%%%%%%%%%%%%%%%%%%%%%%%%%%%%%%%%%%%%%%%%%%
% INTERACTIVE ENVIRONMENT SIMULATOR

go(Initial_Position, Agent) :-
    start_interactive_simulation(Initial_Position, Agent).


%%%%%%%%%%%%%%%%%%%%%%%%%%%%%%%%%%%%%%%%%%%%%%%%%%%%%%%%%%%%%%%%%%%%%%%%%%%%%%%%
start_interactive_simulation(Initial_Position, Agent) :-

    % Clear and initialize the environment internal state
    retractall( agent_position(_)                ),
    retractall( agents_last_action(_)            ),
    retractall( next_percept(_)                  ),
    retractall( path(_, _, _)                    ),
    retractall( camouflage(_)                    ),
    asserta(    agent_position(Initial_Position) ),
    asserta(    agents_last_action(none)         ),
    asserta(    next_percept(init)               ),
    asserta(    path([], 0, [])                  ),
    asserta(    camouflage(off)                  ),

    % Run the simulation
    once(run_interactive_environment(
                                     update_state,
                                     Agent,
                                     termination_function(Reason))
                                    ),
    print('Simulation terminated: '),
    print(Reason),
    nl.

%%%%%%%%%%%%%%%%%%%%%%%%%%%%%%%%%%%%%%%%%%%%%%%%%%%%%%%%%%%%%%%%%%%%%%%%%%%%%%%%
% run_interactive_environment(+Update_Function, +Agent, +Termination_Function)
run_interactive_environment(_Update_Function, _Agent, Termination_Function) :-
    call(Termination_Function),
    show_world,
    print('Termination function succeeded.'),
    nl.

run_interactive_environment(Update_Function, Agent, Termination_Function) :-
    show_world,
    get_percept(P),

    % create and call the agent predicate
    A =.. [Agent, P, Action],
    call(A),

    % create and call the update predicate
    U =.. [Update_Function, Action],
    call(U),

    % I/O
    print_percept(P), nl,
    print_action(Action), nl,
    (
     path(Current_Path, Current_Paths_Cost, [Current_Positions_Frontier|_]) ;
     path(Current_Path, Current_Paths_Cost, Current_Positions_Frontier)
    ),
    print_path(Current_Path, Current_Paths_Cost, Current_Positions_Frontier),
    get_char(_),

    run_interactive_environment(Update_Function, Agent, Termination_Function).

\end{verbatim}

Predicados comunes a los dos entornos de simulaci\'{o}n:

\

La funcion de terminaci\'{o}n contempla todos los casos en los que deber\'{i}a terminar la simulacion por muerte,
victoria o abandono del agente, incluyendo algunas situaciones a las que se sabe que nunca se llegar\'{a} por el
dise\~{n}o de los agentes (por ejemplo, se sabe que un agente nunca morir\'{i}a ahogado por cruzar una celda de agua sin
puente, ya que los agentes fueron dise\~{n}ados de tal manera que s\'{o}lo siguen el camino encontrado por la
b\'{u}squeda, y la b\'{u}squeda no genera celdas con agua sin puente como vecinos v\'{a}lidos).
Esto se implement\'{o} para prop\'{o}sitos de generalidad, y para no asumir nada de los agentes a simular.

\

En los comentarios se detallan las situaciones contempladas por la funci\'{o}n de terminaci\'{o}n.

\begin{verbatim}
%%%%%%%%%%%%%%%%%%%%%%%%%%%%%%%%%%%%%%%%%%%%%%%%%%%%%%%%%%%%%%%%%%%%%%%%%%%%%%%%
% Determines whether the simulation should terminate, and why.
% The simulation terminates if the agent abandons, if the agents dies, or if
% the agents finds the castle.

% The agent abandons
termination_function(abandono) :-
    agents_last_action(abandonar).

% The agent found the castle
termination_function(trompetas) :-
    agent_position(Position),
    celda(Position, pasto, castillo).

% The agent dies if he crosses woods, or water without a bridge, or water with a
% broken bridge.
termination_function(muerto) :-
    agent_position(Position),
    (
     celda(Position, agua, puente(roto)) ;
     celda(Position, agua, -)            ;
     celda(Position, bosque, -)
    ).

% The agent dies if he crosses an intact bridge galloping
termination_function(muerto) :-
    agents_last_action(avanzar(_Direccion, al_galope)),
    agent_position(Position),
    celda(Position, agua, puente(sano)).

% The agent dies if he crosses a deteriorated bridge galloping or at a trot
termination_function(muerto) :-
    (
     agents_last_action(avanzar(Direccion, al_galope)) ;
     agents_last_action(avanzar(Direccion, al_trote))
    ),
    agent_position(Position),
    celda(Position, agua, puente(det)).

% The agent dies if he crosses a friendly army with camouflage on
termination_function(muerto) :-
    camouflage(on),
    agent_position(Position),
    celda(Position, _Terrain, ejercito(reino)).

% The agent dies if he crosses an enemy army with camouflage off
termination_function(muerto) :-
    camouflage(off),
    agent_position(Position),
    celda(Position, _Terrain, ejercito(enemigo)).

\end{verbatim}

El predicado de actualizaci\'{o}n \texttt{update\_state/2} es est\'{a}ndar, actualiza la \'{u}ltima acci\'{o}n del agente
y genera la pr\'{o}xima percepci\'{o}n en base a \'{e}sta.
Las posibles acciones del agente son avanzar, poner o quitarse el disfraz, abandonar, buscar un camino hacia la meta, o
hacer nada.

\begin{verbatim}
%%%%%%%%%%%%%%%%%%%%%%%%%%%%%%%%%%%%%%%%%%%%%%%%%%%%%%%%%%%%%%%%%%%%%%%%%%%%%%%%
% Update predicate for the environment's internal state.
update_state(avanzar(Direccion, Velocidad)) :-

    % Calculate and update the agent's next position based on current position
    % and direction.
    next_position(Direccion, Next_Position),
    valid_position(Next_Position),
    replace(
            agent_position(_),
            agent_position(Next_Position)
           ),
    replace(
            agents_last_action(_),
            agents_last_action(avanzar(Direccion, Velocidad))
           ),

    % Calculate and update the agent's next percept
    percieve(Next_Position, Next_Percept),
    replace(next_percept(_), next_percept(Next_Percept)).

update_state(poner_disfraz) :-
    replace(agents_last_action(_), agents_last_action(poner_disfraz)).

update_state(quitar_disfraz) :-
    replace(agents_last_action(_), agents_last_action(quitar_disfraz)).

update_state(abandonar) :-
    replace(agents_last_action(_), agents_last_action(abandonar)).

update_state(buscando) :-
    % Calculate and update the agent's next percept
    agent_position(Position),
    percieve(Position, Next_Percept),
    replace(next_percept(_), next_percept(Next_Percept)),
    replace(agents_last_action(_), agents_last_action(buscando)).

update_state(none) :-
    % Calculate and update the agent's next percept
    agent_position(Position),
    percieve(Position, Next_Percept),
    replace(next_percept(_), next_percept(Next_Percept)),
    replace(agents_last_action(_), agents_last_action(none)).

\end{verbatim}

El predicado \texttt{next\_position/2} es un predicado de utilidad que dada una direcci\'{o}n de avanze, genera la
pr\'{o}xima posici\'{o}n.
La posici\'{o}n actual la recupera del estado interno del simulador mediante el predicado din\'{a}mico
\texttt{agents\_position/1}.

\begin{verbatim}
%%%%%%%%%%%%%%%%%%%%%%%%%%%%%%%%%%%%%%%%%%%%%%%%%%%%%%%%%%%%%%%%%%%%%%%%%%%%%%%%
next_position(norte, [R1, C]) :-
    agent_position([R, C]),
    R1 is R - 1.
next_position(este, [R, C1]) :-
    agent_position([R, C]),
    C1 is C + 1.
next_position(sur, [R1, C]) :-
    agent_position([R, C]),
    R1 is R + 1.
next_position(oeste, [R, C1]) :-
    agent_position([R, C]),
    C1 is C - 1.

\end{verbatim}

El predicado \texttt{percieve/2} genera una percepci\'{o}n dada una posici\'{o}n.
La percepci\'{o}n generada corresponde a lo que el agente podr\'{i}a percibir desde esa posici\'{o}n.

\begin{verbatim}
%%%%%%%%%%%%%%%%%%%%%%%%%%%%%%%%%%%%%%%%%%%%%%%%%%%%%%%%%%%%%%%%%%%%%%%%%%%%%%%%
percieve([R, C], Percept) :-
    % Generate perception positions
    NR is R - 1,
    NC is C,
    ER is R,
    EC is C + 1,
    SR is R + 1,
    SC is C,
    WR is R,
    WC is C - 1,

    % Get perceptions from world
    celda([NR, NC], North_Terrain, North_Contents),
    celda([SR, SC], South_Terrain, South_Contents),
    celda([ER, EC], East_Terrain,  East_Contents),
    celda([WR, WC], West_Terrain,  West_Contents),

    % Generate the perception
    AlNorte = [[NR, NC], North_Terrain, North_Contents],
    AlSur   = [[SR, SC], South_Terrain, South_Contents],
    AlEste  = [[ER, EC], East_Terrain,  East_Contents],
    AlOeste = [[WR, WC], West_Terrain,  West_Contents],
    Percept = perc([R, C], AlNorte, AlSur, AlEste, AlOeste).

\end{verbatim}

El predicado \texttt{valid\_position/1} es verdadero si la posici\'{o}n suministrada es una posici\'{o}n v\'{a}lida
para el agente.

\begin{verbatim}
%%%%%%%%%%%%%%%%%%%%%%%%%%%%%%%%%%%%%%%%%%%%%%%%%%%%%%%%%%%%%%%%%%%%%%%%%%%%%%%%
valid_position(Position) :-
    (
     celda(Position, pasto, Contents) ;
     celda(Position, mont,  Contents) ;
     celda(Position, agua,  puente(sano)) ;
     celda(Position, agua,  puente(det))
    ).

\end{verbatim}

\section{Agentes}

Los predicados correspondientes a los dos agentes delegan la inteligencia al predicado \newline
\texttt{choose\_action/3}, el cual elige la acci\'{o}n correspondiente en base al par\'{a}metro que contiene al nombre
del agente invocador.
Se implement\'{o} de esta manera ya que los dos agentes deciden tomar la misma acci\'{o}n en todos los casos salvo al
momento de buscar un camino, en cuyo caso cada uno utiliza su estrategia de b\'{u}squeda correspondiente.

\subsection{\texttt{agentedfs.pl}}

\begin{verbatim}
%%%%%%%%%%%%%%%%%%%%%%%%%%%%%%%%%%%%%%%%%%%%%%%%%%%%%%%%%%%%%%%%%%%%%%%%%%%%%%%
% DFS AGENT

dfs(Percept, Action) :-
    choose_action(dfs, Percept, Action).

\end{verbatim}

\subsection{\texttt{agenteastar.pl}}

\begin{verbatim}
%%%%%%%%%%%%%%%%%%%%%%%%%%%%%%%%%%%%%%%%%%%%%%%%%%%%%%%%%%%%%%%%%%%%%%%%%%%%%%%
% A* AGENT

astar(Percept, Action) :-
    choose_action(astar, Percept, Action).

\end{verbatim}

\subsection{\texttt{agentpred.pl}}

El archivo \texttt{agentpred.pl} contiene los predicados comunes a los dos agentes; notablemente,
\texttt{choose\_action/3} y los predicados que utiliza.

El agente mantiene un estado interno propio, el cual consiste de su mapa (\texttt{mapa/3}), un flag indicando si tiene
puesto el disfraz o no (\texttt{camouflage/1}), y el camino hacia la meta e informaci\'{o}n asociada (\texttt{path/3}).
Este en particular merece menci\'{o}n especial.
El primer argumento corresponde a la lista de posiciones en el camino hacia la meta, la meta siendo la \'{u}ltima
posici\'{o}n de la lista y la pr\'{o}xima posici\'{o}n la primera, en la cabeza.
El segundo argumento corresponde al costo del camino actual, seg\'{u}n lo indicado por el algoritmo de b\'{u}squeda, y el
tercero a una lista de listas, cada elemento siendo una frontera correspondiente a la frontera obtenida para ese nodo
durante la ejecuci\'{o}n del algoritmo de b\'{u}squeda.
Esta informaci\'{o}n se almacena para poder mostrarla mediante la interfaz gr\'{a}fica o textual.

\begin{verbatim}
% AGENT'S INTERNAL STATE
:- dynamic mapa/3.               % The agent's map.
:- dynamic camouflage/1.         % camouflage(on) indicates the agent has the
                                 % enemy's attire donned, camouflage(off)
                                 % indicates otherwise.
:- dynamic path/3.               % 1st argument: the path itself, list of
                                 % positions to reach the goal
                                 % 2nd argument: cost of the path
                                 % 3rd argument: list of frontiers, the i-th
                                 % frontier corresponds to the i-th position in
                                 % the path

\end{verbatim}

\texttt{choose\_action/3} es el predicado principal de la l\'{o}gica de los agentes.
Este predicado computa la siguiente acci\'{o}n a tomar segun lo que se percibe.
Los comentarios detallancada caso.

\begin{verbatim}
%%%%%%%%%%%%%%%%%%%%%%%%%%%%%%%%%%%%%%%%%%%%%%%%%%%%%%%%%%%%%%%%%%%%%%%%%%%%%%%
% choose_action(+Agent, +Percept, -Action)
% chooses the corresponding action for each agent

% initial action, search for a path to the castle
choose_action(Agent, init, buscando) :-
    agent_position(Position),
    search(Agent,Position, Solution, Cost, Frontier_List),
    replace(path(_,_,_), path(Solution, Cost, Frontier_List)).

% if the simulation terminated, do nothing.
choose_action(_Agent, fin_simulacion(_Razon), none).

% if an enemy army is encountered, put on the camouflage and don't consume
% the next position
choose_action(_Agent, Percept, poner_disfraz) :-
    path([Next_Position|_], _Cost, _Frontier_List),
    enemy_army_present(Next_Position, Percept),
    camouflage(off),
    replace(camouflage(off), camouflage(on)).

% if an ardos enemy army is encountered, take off the camouflage and
% don't consume the next position
choose_action(_Agent, Percept, quitar_disfraz) :-
    path([Next_Position|_], _Cost, _Frontier_List),
    ardos_army_present(Next_Position, Percept),
    camouflage(on),
    replace(camouflage(on), camouflage(off)).

% if a broken bridge is encountered along the path, begin a new search
% and follow the new path
choose_action(Agent, Percept, buscando) :-
    path([Next_Position|_], _Cost, _Frontier_List),
    broken_bridge(Next_Position, Percept),

    % Update map
    replace(mapa(Next_Position, agua, puente(sano)), mapa(Next_Position, agua, -)),

    % Perform new search
    agent_position(Current_Position),
    search(Agent, Current_Position, New_Path, New_Cost, New_Frontier_List),
    replace(path(_,_,_), path(New_Path, New_Cost, New_Frontier_List)).

% if the search returned an empty path, abandon
choose_action(_Agent, _Percept, abandonar) :-
    path([], 0, []).

% if nothing special is encountered, advance along the path
choose_action(_Agent, Percept, avanzar(Direction, Speed)) :-
    path([Next_Position|Remaining_Path], Cost, [_|Remaining_Frontier_List]),
    choose_direction(Next_Position, Percept, Direction),
    choose_speed(Next_Position, Percept, Speed),
    replace(path(_,_,_), path(Remaining_Path, Cost, Remaining_Frontier_List)).

\end{verbatim}

\texttt{search/5} realiza la b\'{u}squeda correspondiente.

\begin{verbatim}
%%%%%%%%%%%%%%%%%%%%%%%%%%%%%%%%%%%%%%%%%%%%%%%%%%%%%%%%%%%%%%%%%%%%%%%%%%%%%%%
% performs the corresponding search for each agent
search(dfs, Position, Solution, Cost, Frontier_List) :-
     dfs_search(Position, Solution, Cost, Frontier_List).
search(astar, Position, Solution, Cost, Frontier_List) :-
     astar_search(Position, Solution, Cost, Frontier_List).

\end{verbatim}

\texttt{enemy\_army\_present/2} se utiliza para verificar si un ejercito del bando enemigo esta presente en la
siguiente posici\'{o}n. Para esto se vale del predicado de utilidad \texttt{get\_next\_position\_percept/3}, el cual
recupera el componente de la percepci\'{o}n correspondiente a la posici\'{o}n sobre la cual se desea avanzar.

\begin{verbatim}
%%%%%%%%%%%%%%%%%%%%%%%%%%%%%%%%%%%%%%%%%%%%%%%%%%%%%%%%%%%%%%%%%%%%%%%%%%%%%%%
% True if there is an enemy army present in the next position
enemy_army_present(Next_Position, Percept) :-
    get_next_position_percept(Next_Position,
                              Percept,
                              [Next_Position, _, ejercito(enemigo)]).

\end{verbatim}

Idem anterior, para el caso de un ejercito del reino de Ardos.

\begin{verbatim}
%%%%%%%%%%%%%%%%%%%%%%%%%%%%%%%%%%%%%%%%%%%%%%%%%%%%%%%%%%%%%%%%%%%%%%%%%%%%%%%
% True if there is an ardos army present in the next position
ardos_army_present(Next_Position, Percept) :-
    get_next_position_percept(Next_Position,
                              Percept,
                              [Next_Position, _, ejercito(reino)]).

\end{verbatim}

Genera el \'{a}tomo representando la direcci\'{o}n en la cual se desea avanzar a partir de la posici\'{o}n suministrada
(en la pr\'{a}ctica obtenida de la cabeza del camino almacenado en \texttt{path/3}) y la percepci\'{o}n actual.

\begin{verbatim}
%%%%%%%%%%%%%%%%%%%%%%%%%%%%%%%%%%%%%%%%%%%%%%%%%%%%%%%%%%%%%%%%%%%%%%%%%%%%%%%
% compares the position in the path with the percept to decide in which
% direction to advance
choose_direction(P, perc(_,[P,_,_],[_,_,_],[_,_,_],[_,_,_]), norte).
choose_direction(P, perc(_,[_,_,_],[P,_,_],[_,_,_],[_,_,_]), sur).
choose_direction(P, perc(_,[_,_,_],[_,_,_],[P,_,_],[_,_,_]), este).
choose_direction(P, perc(_,[_,_,_],[_,_,_],[_,_,_],[P,_,_]), oeste).

\end{verbatim}

Decide a que velocidad avanzar sobre el cuadro, bas\'{a}ndose en la pr\'{o}xima posici\'{o}n y la percepci\'{o}n.

\begin{verbatim}
%%%%%%%%%%%%%%%%%%%%%%%%%%%%%%%%%%%%%%%%%%%%%%%%%%%%%%%%%%%%%%%%%%%%%%%%%%%%%%%
% Chooses the speed with which to advance upon the next position
choose_speed(Next_Position, Percept, Speed) :-
   get_next_position_percept(Next_Position,
                             Percept,
                             [Next_Position, Terrain, Contents]),
   get_speed(Terrain, Contents, Speed).

\end{verbatim}

Utilizado por el predicado anterior, decide velocidad en base al terreno y contenidos percibidos.

\begin{verbatim}
%%%%%%%%%%%%%%%%%%%%%%%%%%%%%%%%%%%%%%%%%%%%%%%%%%%%%%%%%%%%%%%%%%%%%%%%%%%%%%%
% Determines speed according to terrain and contents
get_speed(mont,  _Contens,     a_paso_hombre).
get_speed(agua,  puente(sano), al_trote).
get_speed(agua,  puente(det),  a_paso_hombre).
get_speed(pasto, _Contents,    al_galope).
get_speed(_,     _,            a_paso_hombre). % just in case.

\end{verbatim}

Testeo de condici\'{o}n semejante a \texttt{enemy\_army\_present/2} y \texttt{ardos\_army\_present/2} para el caso de un
puente roto.

\begin{verbatim}
%%%%%%%%%%%%%%%%%%%%%%%%%%%%%%%%%%%%%%%%%%%%%%%%%%%%%%%%%%%%%%%%%%%%%%%%%%%%%%%
% Determines whether a bridge is broken in the next position
broken_bridge(Next_Position, Percept) :-
    get_next_position_percept(Next_Position,
                              Percept,
                              [Next_Position, agua, puente(roto)]).

\end{verbatim}

\texttt{get\_next\_position\_percept/3} recupera, a partir de una percepci\'{o}n y una posici\'{o}n, la componente de la
estructura \texttt{perc/5} correspondiente a la posici\'{o}n.

\begin{verbatim}
%%%%%%%%%%%%%%%%%%%%%%%%%%%%%%%%%%%%%%%%%%%%%%%%%%%%%%%%%%%%%%%%%%%%%%%%%%%%%%%
% Gets the percept component corresponding to the next position
% NP: north position, NT: north terrain, NC: north contents
% SP: south position, ST: south terrain, SC: south contents
% EP: east position,  ET: east terrain,  EC: east contents
% WP: west position,  WT: west terrain,  WC: west contents
get_next_position_percept(NP,
                          perc(_Pos,
                               [ NP, NT, NC],
                               [_SP,_ST,_SC],
                               [_EP,_ET,_EC],
                               [_WP,_WT,_WC]),
                          [NP, NT, NC]).
get_next_position_percept(SP,
                          perc(_Pos,
                               [_NP,_NT,_NC],
                               [ SP, ST, SC],
                               [_EP,_ET,_EC],
                               [_WP,_WT,_WC]),
                          [SP, ST, SC]).
get_next_position_percept(EP,
                          perc(_Pos,
                               [_NP,_NT,_NC],
                               [_SP,_ST,_SC],
                               [ EP, ET, EC],
                               [_WP,_WT,_WC]),
                          [EP, ET, EC]).
get_next_position_percept(WP,
                          perc(_Pos,
                               [_NP,_NT,_NC],
                               [_SP,_ST,_SC],
                               [_EP,_ET,_EC],
                               [ WP, WT, WC]),
                          [WP, WT, WC]).

\end{verbatim}

\section{B\'{u}squeda}

\subsection{\texttt{search.pl}}

Los siguientes son predicados comunes a ambas estrategias de b\'{u}squeda y los que definen el grafo de exploraci\'{o}n.

\

\texttt{is\_goal/1} es verdadero si la posicion suministrada contiene el castillo.
Notese que esto se verifica mediante \texttt{mapa/3} y no \texttt{celda/3}.

\begin{verbatim}
% Search Graph Predicates:

%%%%%%%%%%%%%%%%%%%%%%%%%%%%%%%%%%%%%%%%%%%%%%%%%%%%%%%%%%%%%%%%%%%%%%%%%%%%%%%
% is_goal([X, Y]) is true if [X, Y] corresponds to a position on the map with
% grass and a castle.
is_goal([R, C]) :-
    mapa([R, C], pasto, castillo).

\end{verbatim}

\texttt{neighbours/4} genera los vecinos para un nodo. Para esto se vale del predicado \texttt{arc/2}, el cual genera
los nodos que cumplen con las condiciones de adyacencia en la grilla.

\begin{verbatim}
%%%%%%%%%%%%%%%%%%%%%%%%%%%%%%%%%%%%%%%%%%%%%%%%%%%%%%%%%%%%%%%%%%%%%%%%%%%%%%%
% neigbhours/4 defines all valid neighbours not in the Frontier or Visited lists.
% neighbours(+Node, +Frontier, +Visited, -Neighbours)
% Neighbours is a list of the valid neighbours of Node
neighbours(node(Position, Path, Path_Cost), Frontier, Visited, Neighbours) :-
    findall(
        node(Neighbour_Positions, [Position|Path], Path_Cost),
        (
            arc(Position, Neighbour_Positions),
            not(member(node(Neighbour_Positions, _, _), Visited)),
            not(member(node(Neighbour_Positions, _, _), Frontier))
        ),
        Neighbours
    ).

\end{verbatim}

Este predicado es utilizado por la versi\'{o}n del algoritmo gen\'{e}rico de b\'{u}squeda que mantiene registro del
costo acumulado del camino. Su prop\'{o}sito es preparar los nodos vecinos para ser agregados a la frontera,
sum\'{a}ndoles el costo del nodo actual al costo del camino desde el origen hasta ellos. Cuando se llegue al nodo meta
el costo calculado ser\'{a} el costo total del camino.

\begin{verbatim}
%%%%%%%%%%%%%%%%%%%%%%%%%%%%%%%%%%%%%%%%%%%%%%%%%%%%%%%%%%%%%%%%%%%%%%%%%%%%%%%
% add paths prepares the neighbours for addition to the frontier by adding the
% cost of the path to the current node to the Path_Cost component of their
% node/3 structure.
add_paths([], _Frontier_Element, []).
add_paths([node(Neighbour, _, _)|Rest_Of_Neighbours],
          node(Position, Path, Path_Cost),
          [node(Neighbour, [Position|Path], New_Path_Cost)|Rest_Of_Frontier]) :-

    cost(Neighbour, Cost),
    New_Path_Cost is Path_Cost + Cost,
    add_paths(Rest_Of_Neighbours, node(Position, Path, Path_Cost), Rest_Of_Frontier).

\end{verbatim}

\texttt{arc/3} genera las posiciones a partir de una posicion dada para las cuales es valido trasitar.
Solo genera un arco si el mapa indica que la celda adyacente tiene agua con un puente sano o no tiene ni agua ni bosque.

\begin{verbatim}
%%%%%%%%%%%%%%%%%%%%%%%%%%%%%%%%%%%%%%%%%%%%%%%%%%%%%%%%%%%%%%%%%%%%%%%%%%%%%%%
% arc([R1, C1], [R2, C2]) is true for all [R2, C2] corresponding to a position
% on the map adjacent to [R1, C2], that is either water with a healthy bridge
% or non-forest. The procedure predicates are ordered in such a way that
% neighbours will be generated in the order North, East, South, West

% To the north
arc([R1, C1], [R2, C2]) :-
    R1 > 0,
    R2 is R1 - 1,
    C2 is C1,
    mapa([R2, C2], Terrain, Contents),
    (
     Terrain = agua, Contents = puente(sano) ;
     Terrain \= agua, Terrain \= bosque
    ).

% To the east
arc([R1, C1], [R2, C2]) :-
    max_column(C),
    C1 < C,
    R2 is R1,
    C2 is C1 + 1,
    mapa([R2, C2], Terrain, Contents),
    (
     Terrain = agua, Contents = puente(sano) ;
     Terrain \= agua, Terrain \= bosque
    ).

% To the south
arc([R1, C1], [R2, C2]) :-
    max_row(R),
    R1 < R,
    R2 is R1 + 1,
    C2 is C1,
    mapa([R2, C2],Terrain,Contents),
    (
     Terrain = agua, Contents = puente(sano) ;
     Terrain \= agua, Terrain \= bosque
    ).

% to the west
arc([R1, C1], [R2, C2]) :-
    C1 > 0,
    R2 is R1,
    C2 is C1 - 1,
    mapa([R2, C2],Terrain,Contents),
    (
     Terrain = agua, Contents = puente(sano) ;
     Terrain \= agua, Terrain \= bosque
    ).

\end{verbatim}

\texttt{cost/2} indica el costo de atravesar una celda, segun su terreno y contenido.

\begin{verbatim}
%%%%%%%%%%%%%%%%%%%%%%%%%%%%%%%%%%%%%%%%%%%%%%%%%%%%%%%%%%%%%%%%%%%%%%%%%%%%%%%
% cost/2 defines the cost of crossing the node corresponding to Position
cost(Position, 1) :-
    mapa(Position, pasto, _Contents).
cost(Position, 2) :-
    mapa(Position, agua, _Contents).
cost(Position, 3) :-
    mapa(Position, mont, _Contents).

\end{verbatim}

El predicado \texttt{frontier\_positions/2} genera una lista de posiciones a partirde una lista de nodos.
Esencialmente elimina la estructura \texttt{node/3} y se queda con la posici\'{o}n.
Este predicado se implement\'{o} porque la estructura \texttt{node/3} es una estructura de ``contabilidad'' propia del
algoritmo de b\'{u}squeda y no del agente o simulador.

\begin{verbatim}
%%%%%%%%%%%%%%%%%%%%%%%%%%%%%%%%%%%%%%%%%%%%%%%%%%%%%%%%%%%%%%%%%%%%%%%%%%%%%%%
% frontier_positions(+Frontier, -Frontier_Positions) produces a list of
% Positions from a Frontier, stripping the Positions of the Path and Path_Cost
% components and the surrounding node/3 bookkeeping structure.
frontier_positions([], []).
frontier_positions([node(N, _P, _PC)|RF], [N|RFP]) :-
    frontier_positions(RF, RFP).

\end{verbatim}

\subsection{Busqueda DFS}

\begin{verbatim}
% DFS SEARCH

%%%%%%%%%%%%%%%%%%%%%%%%%%%%%%%%%%%%%%%%%%%%%%%%%%%%%%%%%%%%%%%%%%%%%%%%%%%%%%%
% dfs_search/4
dfs_search(Position, Solution, Cost, Frontier_List) :-
    dfs_search([node(Position, [], 0)], [], Solution1, Cost, Frontier_List),
    reverse(Solution1, [Position|Solution]).
dfs_search(_Position, [], 0, []).

\end{verbatim}

\begin{verbatim}
%%%%%%%%%%%%%%%%%%%%%%%%%%%%%%%%%%%%%%%%%%%%%%%%%%%%%%%%%%%%%%%%%%%%%%%%%%%%%%%
% dfs_search/5
dfs_search(Frontier,
           _Visited,
           [Position|Path],
           Path_Cost,
           [Frontier_Positions]) :-

    dfs_select(node(Position, Path, Path_Cost), Frontier, _Frontier1),
    is_goal(Position),
    frontier_positions(Frontier, Frontier_Positions).

dfs_search(Frontier,
           Visited,
           Solution,
           Cost,
           [Frontier_Positions|Frontier_List]) :-

    dfs_select(node(Position, Path, Path_Cost), Frontier, Frontier1),
    neighbours(node(Position, Path, Path_Cost), Frontier, Visited, Neighbours),
    add_paths(Neighbours, node(Position, Path, Path_Cost), New_Frontier_Elements),
    dfs_add_to_frontier(New_Frontier_Elements, Frontier1, New_Frontier),
    frontier_positions(New_Frontier, Frontier_Positions),
    dfs_search(New_Frontier,
               [node(Position, Path, Path_Cost)|Visited],
               Solution,
               Cost,
               Frontier_List).

\end{verbatim}

\begin{verbatim}
%%%%%%%%%%%%%%%%%%%%%%%%%%%%%%%%%%%%%%%%%%%%%%%%%%%%%%%%%%%%%%%%%%%%%%%%%%%%%%%
% dfs_select/3
dfs_select(Node, [Node|Frontier], Frontier).

\end{verbatim}

\begin{verbatim}
%%%%%%%%%%%%%%%%%%%%%%%%%%%%%%%%%%%%%%%%%%%%%%%%%%%%%%%%%%%%%%%%%%%%%%%%%%%%%%%
% dfs_add_to_frontier/3
dfs_add_to_frontier(Neighbours, Frontier1, New_Frontier) :-
    append(Neighbours, Frontier1, New_Frontier).

\end{verbatim}

\subsection{Busqueda $A^{*}$}

\begin{verbatim}
% A* SEARCH

%%%%%%%%%%%%%%%%%%%%%%%%%%%%%%%%%%%%%%%%%%%%%%%%%%%%%%%%%%%%%%%%%%%%%%%%%%%%%%%
% astar_search/4
astar_search(Position, Solution, Cost, Frontier_List) :-
    astar_search([node(Position, [], 0)], [], Solution1, Cost, Frontier_List),
    reverse(Solution1, [Position|Solution]).
astar_search(_Position, [], 0, []).

\end{verbatim}

\begin{verbatim}
%%%%%%%%%%%%%%%%%%%%%%%%%%%%%%%%%%%%%%%%%%%%%%%%%%%%%%%%%%%%%%%%%%%%%%%%%%%%%%%
astar_search(Frontier,
             _Visited,
             [Position|Path],
             Path_Cost,
             [Frontier_Positions]) :-

    astar_select(node(Position, Path, Path_Cost), Frontier, _Frontier1),
    is_goal(Position),
    frontier_positions(Frontier, Frontier_Positions).

astar_search(Frontier,
             Visited,
             Solution,
             Cost,
             [Frontier_Positions|Frontier_List]) :-

    astar_select(node(Position, Path, Path_Cost), Frontier, Frontier1),
    neighbours(node(Position, Path, Path_Cost), Frontier, Visited, Neighbours),
    add_paths(Neighbours, node(Position, Path, Path_Cost), New_Frontier_Elements),
    astar_add_to_frontier(New_Frontier_Elements, Frontier1, New_Frontier),
    frontier_positions(New_Frontier, Frontier_Positions),
    astar_search(New_Frontier,
                 [node(Position, Path, Path_Cost)|Visited],
                 Solution,
                 Cost,
                 Frontier_List).

\end{verbatim}

\begin{verbatim}
%%%%%%%%%%%%%%%%%%%%%%%%%%%%%%%%%%%%%%%%%%%%%%%%%%%%%%%%%%%%%%%%%%%%%%%%%%%%%%%
astar_select(Node, [Node|Frontier], Frontier).

\end{verbatim}

\texttt{astar\_add\_to\_frontier/3} es la version $A^{*}$ del predicado ``agregar''.
Trata a la frontera como una cola con prioridad ordenada seg\'{u}n el valor $f$.

\begin{verbatim}
%%%%%%%%%%%%%%%%%%%%%%%%%%%%%%%%%%%%%%%%%%%%%%%%%%%%%%%%%%%%%%%%%%%%%%%%%%%%%%%
astar_add_to_frontier(Neighbours, Frontier1, Frontier3) :-
    append(Frontier1, Neighbours, Frontier2),
    sort_by_f(Frontier2, Frontier3).

\end{verbatim}

\texttt{sort\_by\_f/2} ordena en roden descendente una lista de nodos seg\'{u}n el valor $f$ de cada uno.
El algoritmo utilizado esta basado en el \texttt{quicksort/2} del libro \textit{The Art Of Prolog} de Sterling y Shapiro.

\begin{verbatim}
%%%%%%%%%%%%%%%%%%%%%%%%%%%%%%%%%%%%%%%%%%%%%%%%%%%%%%%%%%%%%%%%%%%%%%%%%%%%%%%
sort_by_f(Frontier1, Frontier2) :-
    quicksort(Frontier1, Frontier2).

%%%%%%%%%%%%%%%%%%%%%%%%%%%%%%%%%%%%%%%%%%%%%%%%%%%%%%%%%%%%%%%%%%%%%%%%%%%%%%%
quicksort([], []).
quicksort([X|Xs], Ys) :-
    partition(Xs, X, Littles, Bigs),
    quicksort(Littles, Ls),
    quicksort(Bigs, Bs),
    append(Ls, [X|Bs], Ys).

%%%%%%%%%%%%%%%%%%%%%%%%%%%%%%%%%%%%%%%%%%%%%%%%%%%%%%%%%%%%%%%%%%%%%%%%%%%%%%%
partition([], _Y, [], []).
partition([X|Xs], Y, [X|Ls], Bs) :-
    f(X, FX),
    f(Y, FY),
    FX =< FY,
    partition(Xs, Y, Ls, Bs).
partition([X|Xs], Y, Ls, [X|Bs]) :-
    f(X, FX),
    f(Y, FY),
    FX > FY,
    partition(Xs, Y, Ls, Bs).

\end{verbatim}

El predicado \texttt{f/2} calcula el valor $f$ de un nodo, lo suma del costo del camino desde el origen hasta \'{e}l y
el costo estimado desde ese nodo hasta la meta.

\begin{verbatim}
%%%%%%%%%%%%%%%%%%%%%%%%%%%%%%%%%%%%%%%%%%%%%%%%%%%%%%%%%%%%%%%%%%%%%%%%%%%%%%%
f(node(Position, _Path, Path_Cost),F) :-
    heuristic(Position, H),
    F is Path_Cost + H.

\end{verbatim}

Este predicado calcula el costo estimado segun la heur\'{i}stica indicada en el enunciado (\textit{SLD}).

\begin{verbatim}
%%%%%%%%%%%%%%%%%%%%%%%%%%%%%%%%%%%%%%%%%%%%%%%%%%%%%%%%%%%%%%%%%%%%%%%%%%%%%%%
heuristic([R1, C1], H) :-
    is_goal([R2, C2]),
    H is (abs(R2 - R1) + abs(C2 - C1)).

\end{verbatim}

\end{document}
