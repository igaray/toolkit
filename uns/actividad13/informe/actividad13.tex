\documentclass[12pt, a4paper]{article}
\usepackage[utf8]{inputenc}
% \usepackage[spanish]{babel}

\usepackage{array}
\usepackage{tabularx}
\usepackage{lipsum}

\usepackage{amsthm}
\newtheorem{theorem}{Teorema}[section]
\newtheorem{lemma}[theorem]{Lema}

\usepackage{minted}
\newminted{cpp}{}
\newmintinline{cpp}{}

\title{UNS Algoritmos y Complejidad 2021 Actividad 13}
\author{Iñaki Garay}
\date{3 de Julio 2021}

\begin{document}

\maketitle

\begin{itemize}
\item \textit{Grupo}: T
\item \textit{Integrantes}: Iñaki Garay L.U. 67387
\item \textit{Usuario HackerRank}: \texttt{@igarai}
\end{itemize}

\section{Implementación}

\begin{minted}[fontsize=\small]{cpp}
#include <algorithm>
#include <iostream>
#include <limits>
#include <queue>
#include <vector>

using namespace std;

const int INF = numeric_limits<int>::max();

typedef struct max_flow_problem {
    int n, m, s, t;
    vector< vector<int> > residual;
    vector< vector<int> > adj;
} max_flow_problem;

max_flow_problem p;

void input() {
    cin >> p.n >> p.m;
    cin >> p.s >> p.t;
    p.s--;
    p.t--;
    p.residual = vector< vector<int> >(p.n);
    for (int i = 0; i < p.n; i++) {
        p.residual[i] = vector<int>(p.n, 0);
    }
    p.adj = vector< vector<int> >(p.n);
    for (int i = 0; i < p.n; i++) {
        p.adj[i] = vector<int>();
    }
    for (int i = 0; i < p.m; i++) {
        int u, v, c;
        cin >> u >> v >> c;
        u--;
        v--;
        p.residual[u][v] = c;
        if (find(p.adj[u].cbegin(), p.adj[u].cend(), v) == p.adj[u].end()) {
            p.adj[u].push_back(v);
        }
        if (find(p.adj[v].cbegin(), p.adj[v].cend(), u) == p.adj[v].end()) {
            p.adj[v].push_back(u);
        }
    }
}

bool bfs(vector<int> &parent) {
    fill(parent.begin(), parent.end(), -1);
    vector<bool> visited = vector<bool>(p.n, false);

    queue<int> q;
    q.push(p.s);

    while (!q.empty()) {
        int u = q.front();
        q.pop();

        for (int v : p.adj[u]) {
            if (!visited[v] && p.residual[u][v] > 0) {
                parent[v] = u;
                visited[v] = true;
                if (v == p.t) {
                    return true;
                }
                q.push(v);
            }
        }
    }
    return false;
}

int edmonds_karp() {
    int max_flow = 0;
    int path_flow = INF;
    vector<int> parent(p.n);

    while (bfs(parent)) {
		int path_flow = INF;
		for (int v = p.t; v != p.s; v = parent[v]) {
			int u = parent[v];
			path_flow = min(path_flow, p.residual[u][v]);
		}
        for (int v = p.t; v != p.s; v = parent[v]) {
            int u = parent[v];
            p.residual[u][v] -= path_flow;
            p.residual[v][u] += path_flow;
        }
        max_flow += path_flow;
    }
    return max_flow;
}

int main() {
    input();
    cout << edmonds_karp();
    return 0;
}
\end{minted}

\section{Análisis del Tiempo de Ejecución}

El algoritmo de Edmonds-Karp es una instancia del método de Ford-Fulkerson en el cual la búsqueda del camino de aumento $p$ se hace mediante una búsqueda por niveles comenzando por $s$.

El recorrido por niveles permite encontrar los caminos mínimos(en cuanto a cantidad de arcos) desde el origen a cada nodo entonces en la red residual, el camino mínimo ya no existe. Estopermite ajustar la cantidad de iteraciones del ciclo WHILE.

Edmonds-Karp is identical to Ford-Fulkerson except for one very important trait. 
The search order of augmenting paths is well defined. As a refresher from the Ford-Fulkerson wiki, augmenting paths, along with residual graphs, are the two important concepts to understand when finding the max flow of a network.

Augmenting paths are simply any path from the source to the sink that can currently take more flow. 
Over the course of the algorithm, flow is monotonically increased. 
So, there are times when a path from the source to the sink can take on more flow, and that is an augmenting path.

Edmonds-Karp differs from Ford-Fulkerson in that it chooses the next augmenting path using breadth-first search (bfs). 
So, if there are multiple augmenting paths to choose from, Edmonds-Karp will be sure to choose the shortest augmenting path from the source to the sink.

The complexity can be given independently of the maximal flow. The algorithm runs in O(VE2) time, even for irrational capacities. 
The intuition is, that every time we find an augmenting path one of the edges becomes saturated, and the distance from the edge to s will be longer, if it appears later again in an augmenting path. 
And the length of a simple paths is bounded by V

\begin{lemma}
En el algoritmo Edmonds-Karp, si $v \in N - {s,t}$ entonces la distancia mínima \texttt{nivel[v]} en $G_{f}$ no disminuye con cada aumento de flujo.
\end{lemma}

\begin{theorem}
El algoritmo Edmonds-Karp en $G =\langle N,A \rangle$ toma $\O(na)$ iteraciones.
\end{theorem}

Demostracion

\begin{proof}
Se muestra que cada arco $(u,v)$ puede ser crítico (esto es, coincide con la capacidad residual) a lo sumo $n/2-1$ veces.

Sea $i$ la iteración donde $(u,v)$ es crítico, y $j$ la iteración donde $(v,u)$ es crítico, 

luego $nivel_{j}[u] = nivel_{j}[v] + 1 \geq	nivel_{i}[v] + 1 = nivel_{i}[u] + 2$,

usando el lema. 

Con lo que la distancia mínima de $u$ aumenta al menos en 2 entre cada para de iteraciones donde $(u,v)$ es arco crítico.

Y $(n-2)$ es una cota de la máxima distancia mínima.

Entonces como hay $\O(a)$ pares de vértices, no puede haber más de $\O(na)$ iteraciones.
\end{proof}

- como cada iteración (construir el grafo residual, hacer BFS,encontrar el camino de aumento y la capacidad residual, y actualizar el flujo) del algoritmo EK toma de O(a), de acuerdo al teorema anterior el tiempo total de ejecución es de O(na2) 

- se elimina de esta forma la dependencia del tiempo de ejecución en $f$*.

Edmonds-Karp relies on many of the proofs and complexities that were described for Ford-Fulkerson. 

To prove that this implementation runs in $\O(V \cdot E^2)$, two statements must be shown to be true. 

The first is that on each iteration of the algorithm, the shortest path between the source and all other vertices in the residual graph must increase monotonic. 
That is, it is always increasing. 

The second is that total number of flow augmentations is $\O(V \cdot E)$. 

If it is true that the shortest path is always increasing in the residual graph (and therefore decreasing in the original graph), and that there are at most $V \cdot E$ flow augmentations, then the bound on the complexity will be well defined. 

\section{Notas}
De manera anecdotica, hubo dos bugs notables durante la implementacion del algoritmo.

- infinity bug

- bfs flow calculation bug

\end{document}